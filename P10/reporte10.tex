\documentclass[a4paper]{article}

%% Language and font encodings
\usepackage[spanish]{babel}
\usepackage[utf8x]{inputenc}
\usepackage[T1]{fontenc}
\usepackage{listings}
\spanishdecimal{.}


%% Sets page size and margins
\usepackage[a4paper,top=3cm,bottom=2cm,left=3cm,right=3cm,marginparwidth=1.75cm]{geometry}

%% Useful packages
\usepackage{amsmath}
\usepackage{graphicx}
\usepackage[colorinlistoftodos]{todonotes}
\usepackage[colorlinks=true, allcolors=blue]{hyperref}

\title{Práctica 10: algoritmo genético}
\begin{document}
\maketitle

\section{Introducci\'on}
En esta tarea se aborda uno de los problemas más populares y trabajados en la rama de la optimización, se trata de el problema de la mochila, el cual consiste en decidir que objetos colocas o seleccionas para llevar en una mochila, considerando que tiene cierta capacidad de objetos y cuyo objetivo es obtener el máximo beneficio. La tarea diez consiste en paralelizar el código dado, y argumentar estadisticamente que la paralelización efectivamente arrojaba los resultados en menor tiempo que de la forma secuencial.

\section{Par\'ametros de trabajo}
La experiemtnación se 

\section{Modificaciones del código}


\begin{lstlisting}[frame=single]
c=rnorm(n)
m <- (floor(abs(c)*50) + 1)
p <- data.frame(x = rnorm(n), y=rnorm(n),c, m)
tiemx <- c()
tiemy <- c()
\end{lstlisting}



\begin{lstlisting}[frame=single]
for (iter in 1:tmax) {
...
 p$x <- foreach(i = 1:n, .combine=c) %dopar% max(min(p[i,]$x + 
 (delta/p[i,]$m) * f[c(TRUE, FALSE)][i], 1), 0)
 tiemx <- c(tiemx,p$x)
 p$y <- foreach(i = 1:n, .combine=c) %dopar% max(min(p[i,]$y + 
 (delta/p[i,]$m) * f[c(FALSE, TRUE)][i], 1), 0)
 tiemy <- c(tiemy,p$y)
...
}
distancia <- c()
for(i in 1:(n*(tmax))){
distancia <- c(distancia, ((tiemx[i]-tiemx[i+50])^2 + 
(tiemy[i]-tiemy[i+50])^2)^(1/2))
}
\end{lstlisting}


\begin{lstlisting}[frame=single]
total <- data.frame(iteraciones,distancia,masa)
total$masa <- as.factor(total$masa)
library('ggplot2')
png(paste("totalR.png", sep=""), width=700, height=700)
ggplot(data=total,aes(x=masa,y=distancia,fill=iteraciones))+
geom_boxplot()#stat_summary(fun.y=mean,geom="smooth",
aes(group=Tipo,col=Tipo))
graphics.off()

write.csv(total,file="TotalReto.csv")
\end{lstlisting}


\begin{lstlisting}[frame=single]
  png("p9radios.png")
  grafica <- ggplot(p, aes(x=p$x, y=p$y))
  grafica+geom_point(aes(size= p$m,col=colores[p$g+6]))+ 
  xlab("x")+ ylab("y") + 
  labs(color= "carga", size="masa")+
  scale_color_manual(labels=seq(5,-5,-1),values=colores)+
  guides(col= guide_legend(override.aes = list(size=3, stroke=1.5))) +
  scale_size_continuous(breaks=seq(0,0.1,0.01),labels=seq(0,0.1,0.01))
  graphics.off()
\end{lstlisting}




\section{Resultados}






\section{Conclusión}
Se incluyen tres GIF, el primero de ellos, \href{https://github.com/ValeryGomez/SimulacionSistemas/blob/master/P9/MasaIgual.gif}{masaIgual} muestra el movimiento original de las partículas sin ninguna masa que los haga ir más rápido o menos. El segundo GIF que se incluye \href{https://github.com/ValeryGomez/SimulacionSistemas/blob/master/P9/Peso.gif}{peso} se muestran las partículas con sus respectivos pesos pero aún sin visualización de su radio. Y el último GIF \href{https://github.com/ValeryGomez/SimulacionSistemas/blob/master/P9/Radios.gif}{radio} las muestra ya con su respectivo radio, es decir en proporción al tamaño de su masa y su mismo movimiento.

\end{document}