\documentclass[a4paper]{article}

%% Language and font encodings
\usepackage[spanish]{babel}
\usepackage[utf8x]{inputenc}
\usepackage[T1]{fontenc}
\usepackage{listings}

%% Sets page size and margins
\usepackage[a4paper,top=3cm,bottom=2cm,left=3cm,right=3cm,marginparwidth=1.75cm]{geometry}

%% Useful packages
\usepackage{amsmath}
\usepackage{graphicx}
\usepackage[colorinlistoftodos]{todonotes}
\usepackage[colorlinks=true, allcolors=blue]{hyperref}

\title{Práctica 4:Diagramas de Voronoi}
\begin{document}
\maketitle

\begin{abstract}
	Para la siguiente práctica esta bien bonita. 

\end{abstract}

\section{Introducci\'on}

\section{Par\'ametros de Trabajo}



\section{Modificaciones del código}



\lstset{language=R, breaklines=true, basicstyle=\footnotesize}

\begin{lstlisting}[frame=single]
for(num in 1: (detectCores()-1)){
registerDoParallel(makeCluster(num))
.
.
.
salida = paste("p3_t", num, ".png", sep="")
png(salida)
boxplot(ot, it, at)
graphics.off()
}
\end{lstlisting}

\section{Resultados}
A continuación presentaremos los gráficos variando la cantidad de núcleos como se especifica de cada uno en el pie del gráfico.


\subsection{Interpretación}



\end{document}
