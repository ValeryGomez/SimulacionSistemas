\documentclass[a4paper]{article}

%% Language and font encodings
\usepackage[spanish]{babel}
\usepackage[utf8x]{inputenc}
\usepackage[T1]{fontenc}
\usepackage{listings}


%% Sets page size and margins
\usepackage[a4paper,top=3cm,bottom=2cm,left=3cm,right=3cm,marginparwidth=1.75cm]{geometry}

%% Useful packages
\usepackage{amsmath}
\usepackage{graphicx}
\usepackage[colorinlistoftodos]{todonotes}
\usepackage[colorlinks=true, allcolors=blue]{hyperref}

\title{Práctica 8: modelo de urnas}
\begin{document}
\maketitle

\section{Introducci\'on}
La siguiente práctica consiste en paralelizar el código original de la práctica

\section{Par\'ametros de trabajo}


\section{Modificaciones del código}

\begin{lstlisting}[frame=single]
replica <- function(t) {
curr <- c(runif(1, low, high),runif(1, low, high))
best <- curr
for (tiempo in 1:t) {
delta <- runif(1, 0, step)
pnts<- c(curr[1]+delta,curr[2],curr[1]-delta,curr[2],
curr[1],curr[2]+delta,curr[1] ,curr[2] -delta,curr[1],curr[2])
valores <- c(g(pnts[1],pnts[2]),g(pnts[3],pnts[4]),g(pnts[5],pnts[6]),
g(pnts[7],pnts[8]),g(curr[1], curr[2]))
v <- which.max(valores)
curr<- c(pnts[(v*2)-1],pnts[v*2])
if (g(curr[1],curr[2]) > g(best[1],best[2])) {
best <- curr
}
}
return(best)
} 
\end{lstlisting}


\section{Resultados}

\subsection{Interpretación}

 
\section{Reto 1}


\section{Reto 2}


\end{document}